\documentclass[12pt, a4paper]{article}
\usepackage[spanish]{babel}
\usepackage[margin=2.5cm]{geometry}
\usepackage{setspace}
\onehalfspacing
\usepackage{indentfirst}
\setlength{\parindent}{1cm}
\usepackage{tabularx}

\title{\textbf{Planificación Inicial del TFG/TFM}}
\author{Nombre del estudiante \\ Tutor: Nombre del Tutor}
\date{Facultad de Informática - Universidad de Murcia}

\begin{document}
\maketitle

\section*{1. Datos Generales}
\begin{itemize}
    \item \textbf{Título provisional:} [Título del TFG/TFM]
    \item \textbf{Estudiante:} [Nombre del estudiante]
    \item \textbf{Tutor(es):} [Nombre del tutor]
    \item \textbf{Fecha de inicio:} [DD/MM/AAAA]
    \item \textbf{Fecha estimada de finalización:} [DD/MM/AAAA]
    \item \textbf{Tipo de TFG/TFM:} [Desarrollo de software, Investigación, etc.]
\end{itemize}

\section*{2. Objetivos}
\begin{itemize}
    \item \textbf{Objetivo general:} [Descripción breve]
    \item \textbf{Objetivos específicos:}
    \begin{enumerate}
        \item Objetivo 1
        \item Objetivo 2
    \end{enumerate}
\end{itemize}

\section*{3. Metodología}
\begin{itemize}
    \item \textbf{Enfoque principal:} [Ágil, tradicional, híbrido]
    \item \textbf{Herramientas a utilizar:} [Trello, GitHub, LaTeX, etc]
    \item \textbf{Fuentes y referencias clave:} [Libros, artículos, etc]
\end{itemize}

\section*{4. Planificación de Fases}
\begin{tabularx}{\textwidth}{|l|X|l|}
    \hline
    \textbf{Fase} & \textbf{Tareas Principales} & \textbf{Fecha Estimada} \\ \hline
    Estado del Arte & Búsqueda bibliográfica, Análisis de referencias & DD/MM - DD/MM \\ \hline
    Diseño y planificación & Definición del alcance. Selección de herramientas & DD/MM - DD/MM \\ \hline
    Desarrollo/Implementación & Implementación del software/análisis de datos & DD/MM - DD/MM \\ \hline
    Validación y resultados & Pruebas, comparación de enfoques, análisis de resultados & DD/MM - DD/MM \\ \hline
    Redacción de la memoria & Elaboración de la estructura. Revisión con tutor & DD/MM - DD/MM \\ \hline
\end{tabularx}

\section*{5. Identificación de Riesgos y Estrategias de Mitigación}
\begin{tabularx}{\textwidth}{|l|X|}
    \hline
    \textbf{Posible Riesgo} & \textbf{Estrategia de Mitigación} \\ \hline
    Dificultad en la recopilación de información & Planificar búsqueda bibliográfica y usar múltiples bases de datos \\ \hline
    Falta de tiempo para completar tareas & Seguir un cronograma ajustado con hitos semanales \\ \hline
    Problemas con herramientas seleccionadas & Tener alternativas y realizar pruebas previas \\ \hline
\end{tabularx}

\section*{6. Sistema de seguimiento y revisión}
\begin{itemize}
    \item \textbf{Frecuencia de reuniones con tutor:} [Semanal, quincenal, etc]
    \item \textbf{Método de comunicación:} [Correo electrónico, reuniones presenciales, etc]
    \item \textbf{Herramientas de seguimiento:} [Trello, GitHub, etc]
    \item \textbf{Criterios de revisión:}
    \begin{enumerate}
        \item Progreso según cronograma
        \item Calidad del trabajo realizado
        \item Dificultades encontradas y soluciones aplicadas
    \end{enumerate}
\end{itemize}

\end{document}