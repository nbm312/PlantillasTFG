\documentclass[12pt, a4paper]{article}
\usepackage[spanish]{babel}
\usepackage[margin=2.5cm]{geometry}
\usepackage{setspace}
\onehalfspacing
\usepackage{indentfirst}
\setlength{\parindent}{1cm}

\title{\textbf{Desarrollo de Software - TFG/TFM}}
\author{Nombre del estudiante \\ Tutor: Nombre del Tutor}
\date{Facultad de Informática - Universidad de Murcia}

\begin{document}
\maketitle

\section*{1. Requisitos del Sistema}
\subsection*{Funcionales}
\begin{itemize}
    \item RF-1. El sistema debe permitir [función específica].
    \item RF-2. El sistema debe permitir [función específica].
\end{itemize}

\subsection*{No Funcionales}
\begin{itemize}
    \item RNF-1. Tiempo de respuesta menor a 2 segundos.
    \item RNF-2. Interfaz de usuario intuitiva.
\end{itemize}

\section*{2. Diseño del Sistema}
\begin{itemize}
    \item \textbf{Diagramas:} [Diagrama de clases, Diagrama de flujo, etc.]
    \item \textbf{Arquitectura:} [Descripción breve de la arquitectura]
\end{itemize}

\section*{3. Implementación}
\begin{itemize}
    \item \textbf{Lenguaje de programación:} [Lenguaje utilizado]
    \item \textbf{Herramientas:} [Git, Docker, etc.]
    \item \textbf{Estructura del código:} [Breve descripción de la estructura]
\end{itemize}

\section*{4. Desarrollo de Juegos}

\begin{itemize}
    \item \textbf{Motor de juego utilizado}: [Unity, Unreal Engine, Godot, etc]
    \item \textbf{Mecánicas de integración}: [Descripción de las mecánicas del juego]
    \item \textbf{Elementos visuales y sonoros}: [Detalles sobre gráficos y música]
\end{itemize}
\section*{5. Pruebas}
\begin{itemize}
    \item \textbf{Pruebas unitarias:} [Descripción de las pruebas]
    \item \textbf{Pruebas de integración:} [Descripción de las pruebas]
    \item \textbf{Resultados de las pruebas:} [Resultados obtenidos]
\end{itemize}

\section*{6. Documentación Técnica}
\begin{itemize}
    \item \textbf{Manual de usuario:} [Instrucciones para el usuario]
    \item \textbf{Documentación del código:} [Breve descripción]
\end{itemize}
\end{document}