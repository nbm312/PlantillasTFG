\documentclass[12pt, a4paper]{article}
\usepackage[spanish]{babel}
\usepackage[margin=2.5cm]{geometry}
\usepackage{setspace}
\onehalfspacing
\usepackage{indentfirst}
\setlength{\parindent}{1cm}

\title{\textbf{Proyecto Interdisciplinar - TFG/TFM}}
\author{Nombre del estudiante \\ Tutor: Nombre del Tutor}
\date{Facultad de Informática - Universidad de Murcia}

\begin{document}
\maketitle
\section*{1. Contexto}
\begin{itemize}
    \item \textbf{Contexto del problema:} [Explica brevemente el problema o reto que se aborda]
    \begin{itemize}
        \item \textbf{\textit{Ejemplo:}} “El análisis de datos médicos requiere tanto conocimientos informáticos como médicos para identificar patrones diagnósticos relevantes”.
    \end{itemize}
    \item \textbf{Objetivos generales: }[Describe los objetivos principales del proyecto]
    \begin{itemize}
        \item \textbf{\textit{Ejemplo:}}\textbf{ “}Desarrollar una herramienta que combine algoritmos de aprendizaje automático con datos médicos para mejorar la precisión diagnóstica”
    \end{itemize}
    \item \textbf{Motivación detrás del enfoque interdisciplinar: }[Justifica por qué es necesario integrar varias disciplinas]
    \begin{itemize}
        \item \textbf{\textit{Ejemplo:}} “La combinación de informática y medicina permite abordar problemas complejos desde múltiples perspectivas.”

    \end{itemize}

    
\end{itemize}



\section*{2. Áreas Involucradas}
\begin{tabular}{|c|c|c|}
    \hline
    \textbf{Área} & \textbf{Objetivos Específicos} & \textbf{Objetivos Compartidos} \\ \hline
    Informática & Desarrollar un algoritmo eficiente & Crear una herramienta para análisis de datos \\ \hline
    Medicina & Identificar patrones diagnósticos relevantes &  \\ \hline
    [Área 3] & [Objetivo Específico] & [Objetivo Compartido] \\ \hline
\end{tabular}

\section*{3. Planificación Conjunta}
\begin{tabular}{|c|c|c|}
    \hline
    \textbf{Tarea} & \textbf{Responsable} & \textbf{Fecha} \\ \hline
    [Tarea 1] & [Responsable] & DD/MM \\ \hline
    [Tarea 2] & [Responsable] & DD/MM \\ \hline
    [Tarea 3] & [Responsable] & DD/MM \\ \hline
\end{tabular}

\section*{4. Integración de enfoques}
\begin{itemize}
    \item \textbf{Descripción de la integración:} [Explica cómo las diferentes áreas contribuyen al resultado final]
\end{itemize}

\section*{5. Resultados}
\begin{itemize}
    \item \textbf{Resultados obtenidos:} [Descripción]
    \item \textbf{Impacto en las áreas:} [Efectos observados]
\end{itemize}
\end{document}