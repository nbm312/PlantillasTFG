% Plantilla-LaTeX/Memoria-TFG.tex
\documentclass[12pt, a4paper]{article}
\usepackage[spanish]{babel}
\usepackage[margin=2.5cm]{geometry}
\usepackage{graphicx}
\usepackage{hyperref}
\usepackage{setspace}
\onehalfspacing

\begin{document}

\begin{titlepage}
\centering

% Logos de la UMU y Facultad (requiere archivos logo-umu.png y logo-fac.png)
\begin{minipage}{0.4\textwidth}
    \centering
    \includegraphics[width=0.9\textwidth]{images/logo-umu.png}
\end{minipage}
\hfill
\begin{minipage}{0.4\textwidth}
    \centering
    \includegraphics[width=0.9\textwidth]{images/logo-fac.png}
\end{minipage}

\vfill % Espacio para centrar el título verticalmente

{\Huge \textbf{TÍTULO DEL DOCUMENTO DEL TFG}\\}
\vspace{1.5cm}

{\large \textbf{Autor:} Nombre Apellidos\\}
\vspace{0.5cm}
{\large \textbf{Tutor:} Nombre del Tutor\\}
\vspace{2cm}

{\large \textbf{Convocatoria:} XXXX 20XX\\}
\vspace{1cm}

{\large \textbf{Grado en Ingeniería Informática}\\}
\vspace{0.5cm}
{\large \textbf{Curso Académico 20XX/20XX}\\}

\vfill % Espacio para centrar todo mejor

\end{titlepage}

\newpage
% Autorización de empresa (si aplica)
\section*{Autorización}
[Texto de autorización según Anexo V.  \href{https://www.um.es/web/informatica/conoce-la-facultad/normativa}{Normativas. Relativos a Trabajos Fin de Grado y Trabajos Fin de Máster}]

\newpage
% Resumen y Extended Abstract
\section*{Resumen}
  [Resumen en español/inglés]

\section*{Extended Abstract}
[Mínimo 2000 palabras en el idioma opuesto]

\newpage
% Índice automático
\tableofcontents

\newpage
% Cuerpo del documento
\section{Introducción}
[Contexto, motivación y objetivos]
\subsection{Contexto}
\subsection{Motivación}
\subsection{Objetivos}

\section{Background}

\begin{itemize}
    \item Contexto general: [Proporciona una visión global del contexto del proyecto]
    \begin{itemize}
        \item \textit{Ejemplo}: “El uso de aplicaciones móviles ha crecido exponencialmente en los últimos años, pero muchas carecen de personalización avanzada.”
    \end{itemize}
    \item Antecedentes relevantes: [Incluye estudios, tecnologías o conceptos clave relacionados]
    \begin{itemize}
        \item \textit{Ejemplo}: “Según el estudio X, las aplicaciones basadas en IA mejoran la eficiencia en un 30%”.

    \end{itemize}
\end{itemize}   

\section{Estado del Arte}
[Revisión bibliográfica]

\section{Objetivos y Metodología}
[Descripción detallada]

\section{Descripción del Diseño, trabajo realizado, pruebas, resultados, etc}
[Implementación, pruebas y análisis]

\section{Conclusiones}
[Síntesis y trabajo futuro]

% Para utilizar la bibliografía, \cite donde corresponda en el texto
\cite{ejemplo1}
\cite{ejemplo2}
\cite{ejemplo3}

\newpage
% Bibliografía (IEEE o APA)
\bibliographystyle{IEEEtran}
\bibliography{IEEE.bib}

% Anexos (opcional)
\appendix
\section{Anexo I}
[Código, datos, etc.]

\end{document}