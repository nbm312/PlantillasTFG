\documentclass[12pt, a4paper]{article}
\usepackage[spanish]{babel}
\usepackage[margin=2.5cm]{geometry}
\usepackage{setspace}
\onehalfspacing
\usepackage{indentfirst}
\setlength{\parindent}{1cm}

\title{\textbf{Aplicación en Empresa - TFG/TFM}}
\author{Nombre del estudiante \\ Tutor: Nombre del Tutor}
\date{Facultad de Informática - Universidad de Murcia}

\begin{document}
\maketitle

\section*{1. Contexto empresarial}
\begin{itemize}
    \item \textbf{Organización involucrada:} [Nombre de la empresa u organización]
    \item \textbf{Área de aplicación:} [Departamento o área donde se implementará]
\end{itemize}
\section*{2. Requisitos del Cliente}
\begin{itemize}
    \item \textbf{Funcionales:}
    \begin{enumerate}
        \item RF-1. El sistema debe permitir [función específica].
        \item RF-2. El sistema debe permitir [función específica].
    \end{enumerate}

    \item \textbf{No funcionales:}
    \begin{enumerate}
        \item RNF-1. Tiempo de respuesta menor a 2 segundos.
        \item RNF-2. Interfaz de usuario intuitiva.
    \end{enumerate}
\end{itemize}

\begin{itemize}
    \item \textbf{Restricciones del proyecto}: [Tiempo, recursos]
\end{itemize}
\section*{3. Desarrollo de la Solución}
\begin{itemize}
    \item \textbf{Herramientas utilizadas:} [Lenguaje de programación, bases de datos, etc.]
    \item \textbf{Proceso de desarrollo:} [Descripción breve]
\end{itemize}

\section*{4. Pruebas Funcionales}
\begin{itemize}
    \item \textbf{Pruebas realizadas:} [Descripción de las pruebas]
    \item \textbf{Resultados:} [Resultados obtenidos]
\end{itemize}

\section*{5. Retroalimentación del Cliente}
\begin{itemize}
    \item \textbf{Comentarios del cliente:} [Feedback recibido]
    \item \textbf{Cambios propuestos:} [Modificaciones sugeridas]
\end{itemize}
\end{document}